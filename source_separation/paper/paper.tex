% -----------------------------------------------
% Template for SMC 2021
% Adapted from previous SMC paper templates
% -----------------------------------------------
\documentclass{article}
\usepackage{smc2021}
%%%%%%%%%%%%%%%%%%%%%%%% Some useful packages %%%%%%%%%%%%%%%%%%%%%%%%%%%%%%%
%%%%%%%%%%%%%%%%%%%%%%%% See related documentation %%%%%%%%%%%%%%%%%%%%%%%%%%
\usepackage[caption=false, font=footnotesize]{subfig}% Modern replacement for subfigure package
\usepackage{paralist}% extended list environments
\usepackage[figure,table]{hypcap}% hyperref companion
% Enable for Review only, remove for Camera Ready version
\pagewiselinenumbers


% Use this if english is the only language/alphabet used in the document
\usepackage[english]{babel}


% Title.
% ------
\def\papertitle{Paper Title}

% Authors
% Please note that submissions are NOT anonymous, therefore 
% authors' names have to be VISIBLE in your manuscript. 
% Authors are entered as an ordered list, each one can be linked to multiple affiliations using the correct index.
% Available tags for authors are: \firstname \middlename \lastname \generation \originalname \email \orcid
% Available tags for affiliations are: \unit \department \institution \streetaddress \city \state \postcode \country \type
% type can take as value: University, Company, Music, Independent, Other
%
% \author[]{\mbox{\firstname{}\middlename{}\lastname{}\originalname{}\generation{}\email{}\orcid{}}}
% mbox force an author not to be split over multiple lines
\author[1]{\mbox{\firstname{Carmine}\lastname{Cella}}}
\author[1]{\mbox{\firstname{Luke}\lastname{Dzwonczyk}}}

%%Affiliations
\affil[1]{\department{Center for New Music and Audio Technologies}\institution{University of California, Berkeley}\city{Berkeley}\state{California}\country{USA}\affiliationtype{University}}



% Complete setup stage
\completesetup

% Title.
% ------
\title{\papertitle}
% ***************************************** the document starts here ***************
\begin{document}
	%
	\capstartfalse
	\maketitle
	\capstarttrue
	%
	
	\begin{abstract}
		The abstract should be placed at the top left column and should contain about 150-200 words.
	\end{abstract}
	%
	
	\section{Introduction}\label{sec:introduction}
	\begin{itemize}
		\item what is CAO 
		\item why we need source separation 
		\item which type of sounds are useful for us (targets)
		\item What is Orchidea?
	\end{itemize}
	
	\textit{The following (from here until the line) is directly copy/pasted from our last paper, it will need to be changed in order to avoid self-plagiarism}
	
	The development of computational tools to assist and inspire the musical composition process constitutes an important research area known as \emph{Computer-Assisted Composition (CAC)} \cite{FerVic2013, Ari2005}. Within CAC, target-based computer-assisted orchestration is a compelling case of how machine learning can be used for {enhancing} and {assisting} music creativity \cite{Maresz2003}. 


	The approach studied in \cite{Carpentier2010} consists in finding a good orchestration for any given sound by searching combinations of sounds from a database with a multi-objective optimization heuristics and a constraint solver that are jointly optimized. Both the target sound and the sounds in the database are embedded in a feature space defined by a fixed feature function and each generated combination of sounds is evaluated by using a specific metric. This method has been substantially improved in \cite{Cella18, Cella2020} and is implemented in the \emph{Orchidea} toolbox for assisted orchestration (\url{www.orch-idea.org}), currently considered the state-of-the-art system for assisted orchestration.

	 The codebase for this paper can be found at: \url{}.
	 
	 ----------------------------------------------------------
	 
	 \textit{Everything below this line in the introduction is newly written for this paper}
	 
	 Target-based computer-assisted orchestration is the process of creating an orchestral score that best matches an arbitrary target sound given a similarity metric and constraints. It recreates the timbre and harmonics of the target sound using orchestral instruments. 
		
		Source separation is an important addition to improve orchestration as it allows for the orchestration of more complex sounds in a way that makes sense in an orchestral context. Orchestral music often uses multiple layers, so by separating the layers of a target and individually orchestrating them, the resulting solution is improved. \textbf{(@Carmine, can you make sure this paragraph makes sense/is correct?)}
		
		For example, consider an example target that has a continuous drone sound throughout and a bird song playing above this drone. Without source separation, Orchidea would detect each note of the bird song as a new onset and cut the target in time at each new note. However, this segmentation in time would also effect the drone, forcing it to have a new onset each time the bird song changes notes. The resulting orchestration would having multiple onsets for the drone, when really only one onset is needed at the beginning of the drone. When source separation is applied to this target, the drone and bird song would be orchestrated separately. Therefore, the drone could be orchestrated by a single onset of note(s) that lasts for the duration of the drone. At the same time, the bird song could have as many new onsets as needed without affecting the drone. The orchestration would be greatly improved by removing the unnecessary onsets in the orchestration of the drone. This is the effect we hope to achieve through implementing source separation in computer-assisted orchestration.
		
	
	\section{Methodology}\label{sec:methodology}
	\begin{itemize}
		\item compare a number of source separation methods
		\item we use the final orchestration to compare the separation methods
	\end{itemize}
	
	We compare the effectiveness of different source separation methods for the task of orchestration by applying the separation methods to various targets, orchestrating the separated output of the method, and finally comparing this orchestration to the orchestration of the target if no separation was performed \textbf{Diagram needed}. For example, consider a target sound that is a combination of a low droning sound and high-pitched whistle. With separation, these two sounds would be disentangled into two separate sub-targets and each would be individually orchestrated. The separated orchestrations would then be combined to create the solution. This would then be compared to the orchestration if no separation had been performed.
	
	
	\section{Experiments}\label{sec:experiments}
	\begin{itemize}
		\item Full orchestration
		\item Ground truth
		\item NMF
		\item 4 neural models: TDCNN++, TDCNN, Demucs, OpenUnmix
		\item table with results
	\end{itemize}
	
	Targets are separated into four sub-targets. Each sub-target is independently orchestrated with a randomly assigned "sub-orchestra." These orchestrations are then combined to play simultaneously, creating a final orchestrated solution. Then the distance between the target and solution is calculated, giving us a metric to compare the various separation methods.
	
		\subsection{Data}
		We created our own targets as combinations of four sub-targets. The sub-targets come from the NIGENS \cite{NIGENS} and BBC \cite{} databases, and freesound.org \cite{freesound}. We selected a total of 90 samples from these databases, choosing sounds that fit the following criteria: 
		
		\begin{enumerate}
			\item Static sounds in which the harmonic average across time is a fitting representation of the sound
			\item Sounds in which there is at least some pitched content and not only noise
		\end{enumerate}			
		Each target that we used for testing was a combination of 4 randomly chosen source sounds from the group of 90 sounds. During the creation of the targets, the sources were randomly offset in time from the beginning of the target, so that they began playing at different times.
		
		\subsection{Separation Methods}
	
			\subsubsection{Non-negative Matrix Factorization}
			@Max 
			
			\subsubsection{Demucs}
			@Leo/Alejandro			
			
			Demucs \cite{demucs}
			
			\subsubsection{OpenUnmix}
			@Leo/Alejandro
			
			\subsubsection{ConvTasNet}
			ConvTasNet \cite{tdcnn} (TDCNN) is a model based on a temporal convolutional encoder-decoder. It uses the audio waveform as input and performs a series of 1-D convolutions at different resolutions with residual connections in order to generate a pool of masks that are then applied on the embedding of the initial waveform to output the separated targets. A final 1-D convolutional layer is used as a decoder to recover the waveforms of each sub target. This type of architecture replaces RNNs in modelling sequential data, as series of dilateds convolutions are a good way to capture temporal patterns.\\
			
			The purpose of the model was initially to separate multiple speakers from a single-channel audio, and was thus trained on speech datasets. Because of its good performance at speech separation, a pre-trained version of the model based on the settings of the paper was added to the benchmark.
			
			\subsubsection{Universal Sound Separation ConvTasNet}
			This version of ConvTasNet, called TDCNN++, was improved by \cite{tdcnnpp} in order to be used for general sound separation, and not only for speech. The main architecture remains the same as in \cite{tdcnn}, and the changes only affected the masking network. More specifically, feature-wise normalization replaces global normalization, residual connections ranges are extended to improve the flow of information, and learnable scale parameters are introduced to weight the outputs of each residual layer. Those modifications help the model to better extract features without drastically increasing its size.\\
			
			TDCNN++ presents the interesting property of having almost the same architecture as TDCNN but trained with generic sounds. Thus, it should naturally be better suited for our task, and the comparison between those two models in particular can give good insights about the impact of the data type used for training.
	
		\subsection{Testing}
		In order to compare the effectiveness of the separation methods, a full target orchestration and a ground truth orchestration are done for each target. The full target orchestration is the orchestration of the target without any separation performed. The ground truth orchestration takes the four sources that make up the target and orchestrates each one separately, then combines them. This ground truth represents the orchestration that could take place if the separation method was perfect and separated a target exactly into its constituent sources. 
		
		The procedure we used for testing is as follows:
		\begin{enumerate}
			\item The target is created as a combination of four randomly chosen sources, which are offset to begin playing at different times
			\item The full target, without any separation performed, is orchestrated using the entire orchestra
			\item A given separation method is performed on the target, splitting the target into 4 sub-targets
			\item Each sub-target is separately orchestrated using a randomly chosen sub-orchestra that is one quarter the size of the full orchestra, and then each orchestration is combined to play simultaneously
			\item The ground truth orchestration is created by orchestrating the four sources, each with a different sub-orchestra, and then combined to play simultaneously
			\item The distances between target and orchestration are calculated for the full target orchestration, the separated orchestration, and the ground truth orchestration 
		\end{enumerate}	
		
		The orchestrations performed were static orchestrations, meaning a single onset of notes is created for each target, no matter if the target itself has multiple onsets.			
		
		\subsection{Evaluation}
		We compare the effectiveness of different separation methods by comparing how well they work for orchestration. The output of a method is orchestrated, and these orchestrations are compared. A quantitative evaluation is performed through the use of a distance metric that measures the spectral distance between target and solution. 
		
		The distance metric cuts the target and solution into successive frames that are 4,000 samples in length, then calculates the spectral distance as defined in Eqn. \ref{eqn:distance} between corresponding frames. The spectral distance metric is proposed in \cite{Cella2020} as part of the cost function used in Orchidea during the optimization, and used in \cite{Cella2020b} \textbf{(is it weird/bad to cite our last paper?)} to compare accuracies of orchestrations. The equation takes in the full FFT of the target $x$ and full FFT of the solution $\tilde{x}$. Then for each bin $k$ of the FFT, it calculates the absolute difference between the values. The differing values of $\lambda_1$ and $\lambda_2$ allow the metric to penalize the solution in different ways.
		
		\begin{equation}\label{eqn:distance}
d(x, \tilde{x}) =\lambda_1 \sum_k \delta_{k1}(x_k - \tilde{x}_k) + \lambda_2 \sum_k \delta_{k2}|x_k - \tilde{x	}_k| \\
\end{equation}
where $\delta_{k1} = 1 \text{  if  } x_k \ge \tilde{x}_k, 0 \text{  otherwise}$; and $\delta_{k2} = 1 \text{  if  } x_k < \tilde{x}_k, 0 \text{  otherwise}$.
	
	
	\begin{table}[t]
		\begin{center}
			\begin{tabular}{|c|c|}
				\hline
				& Average distance \\
				\hline
				Full target & 25.73 \\
				\hline
				TDCNN++ & 25.44 \\
				\hline
				Demucs & 27.62 \\
				\hline
				NMF & 22.32 \\
				\hline
				Ground truth & 24.27 \\
				\hline
			\end{tabular}
		\end{center}
		\caption{Average across \textbf{X} targets showing the distance between target and orchestration for various methods. "Full target" means no separation was performed. }
		\label{tab:distances}
	\end{table}
	
	\section{Conclusions}\label{sec:conclusions}
	In this paper, we addressed the problem of orchestrating dynamic targets using source separation as a preprocessing step. The results showed that performing source separation followed by a static orchestration of each individual sub targets improved the spectral distance between the original sample and its orchestrated version.\\
	
	Furthermore, it seems that unsupervised methods such as NMF outperform all supervised methods. However, it is important to notice that the different neural models (Demucs, ConvTasNet, ConvTasNetUniversal and OpenUnmix) were used based on pre-trained versions. Those versions were trained on musical data, except for TDCNN++ which used selected generic sound samples, which opens a large path for improvement. More specifically, we may infer from our results that models trained not only on musical data can generate a better separation for orchestration, as TDCNN++ outperforms other supervised models.\\
	
	Since unsupervised methods perform well, it is reasonable to think that supervised methods trained on a dataset well suited for source separation could yield better overall performances on dynamic orchestration.
	
	\section{Future Work}\label{sec:futurework}
	Implementing these methods in Orchidea. We could improve the supervised methods by training them ourselves with data that fits our problem better. Adding a joint optimization to determine which instruments should be used in which sources.
	
	\begin{acknowledgments}
		At the end of the Conclusions, acknowledgements to people, projects, funding agencies, etc. can be included after the second-level heading  ``Acknowledgments'' (with no numbering).
	\end{acknowledgments} 
	
	%%%%%%%%%%%%%%%%%%%%%%%%%%%%%%%%%%%%%%%%%%%%%%%%%%%%%%%%%%%%%%%%%%%%%%%%%%%%%
	%bibliography here
	\bibliography{references}
	
\end{document}
