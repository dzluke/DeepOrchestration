% -----------------------------------------------
% Template for SMC 2021
% Adapted from previous SMC paper templates
% -----------------------------------------------
\documentclass{article}
\usepackage{smc2021}
%%%%%%%%%%%%%%%%%%%%%%%% Some useful packages %%%%%%%%%%%%%%%%%%%%%%%%%%%%%%%
%%%%%%%%%%%%%%%%%%%%%%%% See related documentation %%%%%%%%%%%%%%%%%%%%%%%%%%
\usepackage[caption=false, font=footnotesize]{subfig}% Modern replacement for subfigure package
\usepackage{paralist}% extended list environments
\usepackage[figure,table]{hypcap}% hyperref companion
% Enable for Review only, remove for Camera Ready version
\pagewiselinenumbers


% Use this if english is the only language/alphabet used in the document
\usepackage[english]{babel}


% Title.
% ------
\def\papertitle{Paper Title}

% Authors
% Please note that submissions are NOT anonymous, therefore 
% authors' names have to be VISIBLE in your manuscript. 
% Authors are entered as an ordered list, each one can be linked to multiple affiliations using the correct index.
% Available tags for authors are: \firstname \middlename \lastname \generation \originalname \email \orcid
% Available tags for affiliations are: \unit \department \institution \streetaddress \city \state \postcode \country \type
% type can take as value: University, Company, Music, Independent, Other
%
% \author[]{\mbox{\firstname{}\middlename{}\lastname{}\originalname{}\generation{}\email{}\orcid{}}}
% mbox force an author not to be split over multiple lines
\author[1]{\mbox{\firstname{Carmine}\lastname{Cella}}}

%%Affiliations
\affil[1]{\department{Center for New Music and Audio Technologies}\institution{University of California, Berkeley}\city{Berkeley}\state{California}\country{USA}\affiliationtype{University}}



% Complete setup stage
\completesetup

% Title.
% ------
\title{\papertitle}
% ***************************************** the document starts here ***************
\begin{document}
	%
	\capstartfalse
	\maketitle
	\capstarttrue
	%
	
	\begin{abstract}
		The abstract should be placed at the top left column and should contain about 150-200 words.
	\end{abstract}
	%
	
	\section{Introduction}\label{sec:introduction}
	\begin{itemize}
		\item what is CAO 
		\item why we need source separation 
		\item which type of sounds are useful for us (targets)
	\end{itemize}		
	
	\section{Methodology}\label{sec:methodology}
	We compare a number of source separation methods and our metrics depend on the orchestration
	Distance metric and formula here
	
	\section{Experiments}\label{sec:experiments}
	\begin{itemize}
		\item Full orchestration
		\item Ground truth
		\item NMF
		\item 4 neural models: TDCNN++, TDCNN, Demucs, OpenUnmix
		\item table with results
	\end{itemize}
	
	\begin{table}[t]
		\begin{center}
			\begin{tabular}{|c|c|}
				\hline
				& Average distance \\
				\hline
				Full target & 25.73 \\
				\hline
				TDCNN++ & 25.44 \\
				\hline
				Demucs & 27.62 \\
				\hline
				NMF & 22.32 \\
				\hline
				Ground truth & 24.27 \\
				\hline
			\end{tabular}
		\end{center}
		\caption{Average distance between target and orchestration for various methods. "Full target" means no separation.}
		\label{tab:distances}
	\end{table}
	
	\section{Conclusions}\label{sec:conclusions}
	We think that adding source separation improves orchestration. Unsupervised methods work better because of the data that supervised methods are trained on \cite{Someone:00}
	
	\section{Future Work}\label{sec:futurework}
	Implementing these methods in Orchidea. We could improve the supervised methods by training them ourselves with data that fits our problem better.
	
	\begin{acknowledgments}
		At the end of the Conclusions, acknowledgements to people, projects, funding agencies, etc. can be included after the second-level heading  ``Acknowledgments'' (with no numbering).
	\end{acknowledgments} 
	
	%%%%%%%%%%%%%%%%%%%%%%%%%%%%%%%%%%%%%%%%%%%%%%%%%%%%%%%%%%%%%%%%%%%%%%%%%%%%%
	%bibliography here
	\bibliography{references}
	
\end{document}
