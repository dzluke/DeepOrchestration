% -----------------------------------------------
% Template for ISMIR Papers
% 2020 version, based on previous ISMIR templates

% Requirements :
% * 6+n page length maximum
% * 4MB maximum file size
% * Copyright note must appear in the bottom left corner of first page
% * Clearer statement about citing own work in anonymized submission
% (see conference website for additional details)
% -----------------------------------------------

\documentclass{article}
\usepackage[T1]{fontenc} % add special characters (e.g., umlaute)
\usepackage[utf8]{inputenc} % set utf-8 as default input encoding
\usepackage{ismir,amsmath,cite,url}
\usepackage{graphicx}
\usepackage{color}

% Optional: To use hyperref, uncomment the following.
% \usepackage[bookmarks=false,hidelinks]{hyperref}
% Mind the bookmarks=false option; bookmarks are incompatible with ismir.sty.

\usepackage{lineno}
\linenumbers

% Title.
% ------
\title{Deep learning approaches on musical assisted-orchestration: an evaluation study}

% Note: Please do NOT use \thanks or a \footnote in any of the author markup

% Single address
% To use with only one author or several with the same address
% ---------------
%\oneauthor
% {Names should be omitted for double-blind reviewing}
% {Affiliations should be omitted for double-blind reviewing}

% Two addresses
% --------------
%\twoauthors
%  {First author} {School \\ Department}
%  {Second author} {Company \\ Address}

%% To make customize author list in Creative Common license, uncomment and customize the next line
%  \def\authorname{First Author, Second Author}


% Three addresses
% --------------
\threeauthors
  {First Author} {Affiliation1 \\ {\tt author1@ismir.edu}}
  {Second Author} {\bf Retain these fake authors in\\\bf submission to preserve the formatting}
  {Third Author} {Affiliation3 \\ {\tt author3@ismir.edu}}

%% To make customize author list in Creative Common license, uncomment and customize the next line
%  \def\authorname{First Author, Second Author, Third Author}

% Four or more addresses
% OR alternative format for large number of co-authors
% ------------
%\multauthor
%{First author$^1$ \hspace{1cm} Second author$^1$ \hspace{1cm} Third author$^2$} { \bfseries{Fourth author$^3$ \hspace{1cm} Fifth author$^2$ \hspace{1cm} Sixth author$^1$}\\
%  $^1$ Department of Computer Science, University , Country\\
%$^2$ International Laboratories, City, Country\\
%$^3$  Company, Address\\
%{\tt\small CorrespondenceAuthor@ismir.edu, PossibleOtherAuthor@ismir.edu}
%}
%\def\authorname{First author, Second author, Third author, Fourth author, Fifth author, Sixth author}


\sloppy % please retain sloppy command for improved formatting

\begin{document}

%
\maketitle
%
\begin{abstract}
The abstract should be placed at the top left column and should contain about 150-200 words.
\end{abstract}
%
\section{Introduction}\label{sec:introduction}

\section{Our model}

\subsection{Baseline}
In order to have a baseline to compare our results against, we attempted to solve the classification problem using various linear classifiers. The classifiers we tested were Support Vector Machine (SVM), Random Forest, and K-Nearest Neighbors. We used the implementations provided in the scikit-learn library for each classifier. For SVM, we used SVC with an RBF kernel. For Random Forest, we set the maximum depth of each tree to be 15. Each classifier was wrapped in a MultiOutputClassifier to achieve the multi-label nature of this problem. We found SVM to have the highest accuracy of the three classifiers across all experiments. All of the following baselin experiments used 50,000 generated samples with a train-test split of 60/40. Each sample is a combination of one or more instruments and is four seconds in length. The features used are the mel frequency cepstral coefficients (MFCCs) of the resulting combination, with a total of 100 coefficients per sample.

We started by simplifying the problem to classifying only the instruments and not the pitch. This had the benefit of both reducing the number of classes and increasing the number of samples per class. We found that SVM was able to very accurately identify the instrument given an input that had only one instrument present; for this case the accuracy was 99.8\%. However as soon as multiple instruments were present in the input, the accuracy dropped significantly. With two instruments, accuracy was 55.4\%, with three it was 19.6\% and with 10 instruments the accuracy was 0.5\%. 

To better approximate the problem of identifying instrument and pitch, we then attempted to classify the instrument and pitch class. That is, which octave the pitch was in did not matter, only the pitch class. The input was a combination of two instruments drawn from a possible twelve instruments, and the classifier attempted to identify which instruments were present and for two of those instruments, say Flute and Violin, which pitch classes were present. If another instrument was present that was not Flute or Violin, the classifier would attempt to identify that instrument, but not its pitch class. The best results from this setting of the problem was SVM with 30.2\% accuracy, which was a result of classifying the pitch class of Flute and Violin. Depending on which two instruments had their pitch class identified, the accuracy varied greatly. For a Violin and Cello accuracy was 20.9\%, and for Trombone and Bass Tuba accuracy was 2.2\%.

For a slight modification on this setting, we no longer attempted to identify which instrument was present if it was not one of the two instruments whose pitch class was being identified. Instead, the classifier would simply identify that an instrument that was not one of the two was present. Since this is a simpler problem, it lead to increased accuracies. Flute and Violin went from 30.2\% to 38.8\%, Oboe and French Horn from 27.3\% to 39.9\%, and Trombone and Bass Tuba from 2.2\% to 3.5\%. \textit{(I think we should maybe put a table of this data instead of listing it all out)} Random Forest performed significantly worse for this setting. Flute and Violin had an accuracy of 9.8\% and Oboe and French Horn was 17.5\%. For this reason, we stopped testing with Random Forest and continued with SVM only.

We then performed this same experiment with three instruments having their pitch class identified. Each input was a combination of three instruments drawn from a possible twelve instruments. For three instruments specified in that experiment, pitch class was identified. Flute, Oboe, and Violin reached an accuracy of 11.1\%, and Bass Tuba, Trumpet, and Trombone was 0.5\%. As we increased the number of instruments whose pitch classes was being identified, the accuracy continued to drop. For classifying the pitch class of four instruments: Oboe, French Horn, Violin, and Flute, the accuracy was 2.7\%.

This was still a simplified version of the problem, as we were identifying only the pitch class of a few instruments. However, the linear classifiers were unable to achieve accurate results as the number of instruments increased. Therefore, we did not attempt the full setting of the problem, in which individual pitches are classified for all instruments, with linear classifiers. 


\subsection{CNN with LSTM}

\subsection{ResNet}

\section{Orchestration experiments}
List of experiments:

- CNN with varying n

- ResNet with n=10

\section{Evaluation and Conclusions}

\subsection{Interpreting the Latent Space}

\section{Future steps}


\section{Citations}

All bibliographical references should be listed at the end,
inside a section named ``REFERENCES,'' numbered and in alphabetical order.
All references listed should be cited in the text.
When referring to a document, type the number in square brackets
\cite{Author:00}, or for a range \cite{Author:00,Someone:10,Someone:04}.

When the following words appear in the conference publication titles, please abbreviate them: Proceedings $\rightarrow$ Proc.; Record $\rightarrow$ Rec.; Symposium $\rightarrow$ Symp.; Technical Digest $\rightarrow$ Tech. Dig.; Technical Paper $\rightarrow$ Tech. Paper; First $\rightarrow$ 1st; Second $\rightarrow$ 2nd; Third $\rightarrow$ 3rd; Fourth/nth $\rightarrow$ 4th/nth.

\textcolor{red}{As submission is double blind, refer to your own published work in the third person. That is, use ``In the previous work of \cite{Someone:10},'' not ``In our previous work \cite{Someone:10}.'' If you cite your other papers that are not widely available (e.g., a journal paper under review), use anonymous author names in the citation, e.g., an author of the form ``A. Anonymous.''}

% For bibtex users:
\bibliography{ISMIRtemplate}

% For non bibtex users:
%\begin{thebibliography}{citations}
%
%\bibitem {Author:00}
%E. Author.
%``The Title of the Conference Paper,''
%{\it Proceedings of the International Symposium
%on Music Information Retrieval}, pp.~000--111, 2000.
%
%\bibitem{Someone:10}
%A. Someone, B. Someone, and C. Someone.
%``The Title of the Journal Paper,''
%{\it Journal of New Music Research},
%Vol.~A, No.~B, pp.~111--222, 2010.
%
%\bibitem{Someone:04} X. Someone and Y. Someone. {\it Title of the Book},
%    Editorial Acme, Porto, 2012.
%
%\end{thebibliography}

\end{document}

